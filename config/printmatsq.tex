

% Prints expanded (m x n) matrix with spanning dots (...) 
% USAGE - \printmatsq{x}

\def \printmat #1{
    \begin{pmatrix}
        #1_{11} & \cdots & #1_{1n} \\
        \vdots  & \ddots & \vdots \\
        #1_{m1} & \cdots & #1_{mn} \\
    \end{pmatrix}
}

\def \printmatsq #1{
    \begin{pmatrix}
        #1_{11} & \cdots & #1_{1n} \\
        \vdots  & \ddots & \vdots \\
        #1_{n1} & \cdots & #1_{nn} \\
    \end{pmatrix}
}

\def \printmatrow #1{
    \begin{pNiceMatrix} 
        \Cdots & \boldsymbol{#1}_1 & \Cdots \\ 
          &  \vdots &  \\
        \Cdots & \boldsymbol{#1}_n & \Cdots 
    \end{pNiceMatrix}
}

\def \printmatcol #1{
\begin{pNiceMatrix} 
        \Vdots &  & \Vdots \\ 
        \boldsymbol{#1}_1 & \cdots  & \boldsymbol{#1}_n \\
        \Vdots &  & \Vdots 
\end{pNiceMatrix} 
}

\def \printrowvec #1{
    \begin{pmatrix}
        #1_1 & \cdots & #1_n
    \end{pmatrix}
}

\def \printcolvec #1{
    \begin{pmatrix}
        #1_1 \\ \vdots \\ #1_n
    \end{pmatrix}
}


\def \printcolvecsolid #1{
    \begin{pNiceMatrix} \Vdots \\ \boldsymbol{#1} \\ \Vdots \end{pNiceMatrix}
}