% Cornell style notes (keywords left column and content right column)
\setlength{\parindent}{0pt}
\newcommand{\newentry}{\vskip 15pt}

% Commands for margin notes
\newcommand{\keyword}[1]{\reversemarginpar \marginnote{\textcolor{ucsb_blue}{\textbf{#1}}}}
\newcommand{\defn}[1]{\reversemarginpar \marginnote{\textcolor{ucsb_blue}{\textbf{#1}}}}
\newcommand{\note}[1]{\reversemarginpar \marginnote{\textcolor{ucsb_green}{\textbf{#1}}}}
\newcommand{\excr}[1]{\reversemarginpar \marginnote{\textcolor{ucsb_coral}{\textbf{#1}}}}
\newcommand{\exmpl}[1]{\reversemarginpar \marginnote{\textcolor{ucsb_gold}{\textbf{#1}}}}

\newcommand{\sectionend}{\reversemarginpar \marginnote{\textcolor{red}{\textbf{[SECTION END]}}}}


\newcommand{\dict}[2]{\textbf{#1} \normalmarginpar \marginnote{{\small #1: #2}}}
\newcommand{\keywordfont}{\ttfamily}

% Theorem and proof macros with margin notes
\newcommand{\thm}[1]{\reversemarginpar \marginnote{\textcolor{ucsb_blue}{\textbf{Theorem}}}}
\newcommand{\prf}[1]{\reversemarginpar \marginnote{\textcolor{ucsb_green}{\textbf{Proof}}}}
\newcommand{\lem}[1]{\reversemarginpar \marginnote{\textcolor{ucsb_blue}{\textbf{Lemma}}}}
\newcommand{\cor}[1]{\reversemarginpar \marginnote{\textcolor{ucsb_blue}{\textbf{Corollary}}}}
\newcommand{\prop}[1]{\reversemarginpar \marginnote{\textcolor{ucsb_blue}{\textbf{Proposition}}}}

% Environment-style commands for theorems and proofs
\newcommand{\addThm}[2]{%
    \newentry
    \thm{Theorem}
    \textbf{#1.} #2
}

\newcommand{\addPrf}[1]{%
    \newentry
    \prf{Proof}
    #1 \hfill $\square$
}

\newcommand{\addLem}[2]{%
    \newentry
    \lem{Lemma}
    \textbf{#1.} #2
}

\newcommand{\addCor}[2]{%
    \newentry
    \cor{Corollary}
    \textbf{#1.} #2
}

\newcommand{\addProp}[2]{%
    \newentry
    \prop{Proposition}
    \textbf{#1.} #2
}

\newcommand{\addDef}[2]{%
    \newentry
    \defn{#1}
    #2  
}

\newcommand{\addNote}[1]{%
    \newentry
    \note{NOTE}
    #1
}

\newcommand{\textBox}[1]{%
    \newentry
    #1
}

\newcommand{\addExmpl}[1]{%
    \newentry
    \exmpl{Examples}
    #1
}

\newcommand{\addExcr}[1]{%
    \newentry
    \excr{Exercise}
    #1
}