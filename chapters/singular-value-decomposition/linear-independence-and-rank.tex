\documentclass[../../main.tex]{subfiles}
\begin{document}
\newgeometry{
    left=35mm,
    top=20mm,
    bottom=30mm,
}

\subsection{Linear Independence}

\addDef{Linear Combination}{
    A linear combination of a set of $\fR^{n}$ vectors $\{ \bx_1, \bx_2, \cdots, \bx_m \}$ is given by 
    \begin{equation*}
        \sum_{i=1}^{m} \alpha_i \bx_i,
    \end{equation*}
    where $\alpha_1, \cdots,  \alpha_n$ are scalars. 
}


\addDef{Linear Dependence}{
    A set of vectors $\{ \bx_1, \bx_2, \cdots, \bx_n \}$ are linearly dependent if there exists a linear combination such that
    \begin{equation*}
        \sum_{i=1}^{m} \alpha_i \bx_i = \mathbf{0},
    \end{equation*}
    where, $\mathbf{0}$ is the zero vector
}

\addNote{
    Conversely, a set of vectors are linearly \textit{independent} if $\sum_{i} \alpha_i \bx_i \neq \mathbf{0}$ for all possible scalar values. 
}

\addDef{Span}{ 
    The set of all possible linear combinations of a set of vectors is called the span of the set. 
}



\subsection{Matrix Rank}

Recall that the rows and columns of a matrix can be interpreted as a set of vectors.

$$
    \bA = \printmatsq{a} = \printmatrow{r} = \printmatcol{c}.
$$

\addDef{Rank}{
    The rank of a matrix $\bA \in \fR^{m \times n}$ is the number of linearly independent rows of $\bA$.
    \begin{itemize}
        \item Gaussian elimination 
    \end{itemize}
}


\addDef{Row Space} {
    The row space of a matrix $\bA \in \fR^{m \times n}$ is the \textit{span} of the matrix's row vectors $\{ \mathbf{r}_1, \mathbf{r}_2, \cdots, \mathbf{r}_n \}$
}

\addDef{Column Space}{
    Conversely, the column space would be the \textit{span} of the matrix's column vectors $\{ \mathbf{c}_1, \mathbf{c}_2, \cdots, \mathbf{c}_n \}$
}
\end{document}