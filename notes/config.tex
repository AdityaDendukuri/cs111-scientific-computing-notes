

%% Cornell style notes (keywords left column and content right column)
\newcommand{\keyword}[1]{\reversemarginpar \marginnote{#1}}
\newcommand{\defn}[1]{\reversemarginpar \marginnote{\textcolor{ucsb_blue}{\textbf{#1}}}}
\newcommand{\note}[1]{\reversemarginpar \marginnote{\textcolor{ucsb_green}{\textbf{#1}}}}
\newcommand{\excr}[1]{\reversemarginpar \marginnote{\textcolor{red}{\textbf{#1}}}}

\newcommand{\dict}[2]{#1\normalmarginpar \marginnote{{\small #1: #2}}}
\newcommand{\keywordfont}{\ttfamily}

%% symbol shortcuts
% fancy symbols
\newcommand{\fR}{\mathbb{R}}
\newcommand{\fC}{\mathbb{C}}
\newcommand{\fZ}{\mathbb{Z}}
% bold symbols
\newcommand{\bx}{\mathbf{x}}
\newcommand{\bX}{\mathbf{X}}
\newcommand{\by}{\mathbf{y}}
\newcommand{\bY}{\mathbf{Y}}
\newcommand{\ba}{\mathbf{a}}
\newcommand{\bA}{\mathbf{A}}
\newcommand{\bb}{\mathbf{b}}
\newcommand{\bB}{\mathbf{B}}
\newcommand{\br}{\mathbf{r}}
\newcommand{\bc}{\mathbf{c}}

%% UCSB colors
\definecolor{ucsb_blue}{RGB}{0, 54, 96}
\definecolor{ucsb_green}{RGB}{9, 132, 122}
\definecolor{ucsb_gold}{RGB}{254, 188, 17}

%% Section title colors
%\titleformat{\section}{\LARGE}{\rlap{\color{ucsb_blue}\rule[-0.4cm]{\linewidth}{1.2cm}} \thesection}{1em}{}


\def \printmatsq #1{
    \begin{bmatrix}
        #1_{11} & \cdots & #1_{1n} \\
        \vdots  & \ddots & \vdots \\
        #1_{m1} & \cdots & #1_{mn} \\
    \end{bmatrix}
}

\def \printmatrow #1{
    \begin{bNiceMatrix} 
        \Cdots & \mathbf{#1}_1 & \Cdots \\ 
          &  \vdots &  \\
        \Cdots & \mathbf{#1}_3 & \Cdots 
    \end{bNiceMatrix}
}

\def \printmatcol #1{
\begin{bNiceMatrix} 
        \Vdots &  & \Vdots \\ 
        \mathbf{#1}_1 & \cdots  & \mathbf{#1}_n \\
        \Vdots &  & \Vdots 
\end{bNiceMatrix} 
}

\def \printrowvec #1{
    \begin{bmatrix}
        #1_1 & \cdots & #1_n
    \end{bmatrix}
}

\def \printcolvec #1{
    \begin{bmatrix}
        #1_1 \\ \vdots \\ #1_n
    \end{bmatrix}
}